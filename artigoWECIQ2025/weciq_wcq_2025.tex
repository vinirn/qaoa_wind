%% ------------- Portuguese version ------------
\documentclass{weciq}
\usepackage[english,brazil]{babel}
\usepackage[utf8]{inputenc}
\newtheorem{theorem}{Teorema}
%% ---------------------------------------------

%% If writing in English, remove the lines above
%% and uncomment the lines below

%% ------------- English version ---------------
%\documentclass[english]{sbrt}
%\usepackage[english]{babel}
%\usepackage[utf8]{inputenc}
%\newtheorem{theorem}{Theorem}
%% ---------------------------------------------

\begin{document}

\newcommand{\weciqwcq}{VIII $\left<\right.$WECIQ$|$WCQ$\left.\right>$}

\title{Modelo de Artigo para o \weciqwcq}

\author{Nome1 Sobrenome1 e Nome2 Sobrenome2
\thanks{Nome1 Sobrenome1, Departamento1, Universidade1, Cidade1-UF1, e-mail: xxxxx@yyyyy.zzzzz.br; Nome2 Sobrenome2, Departamento2, Universidade2, Cidade2-UF2, e-mail: xxxxx@yyyyy.zzzzz.br. Este trabalho foi parcialmente financiado por XXXXXXX (XX/XXXXX-X).}%
}

\maketitle

\markboth{VIII Workshop-Escola de Computação e Informação Quântica \& VIII Workshop de Computação Quântica - UFSC, 08--12 de Dezembro de 2025, Florianópolis, SC}{}


%% If writing in English, remove both 'resumo' and 'chave'
%% ------------- Portuguese version ------------
\begin{resumo}
Este documento apresenta um exemplo de utilização do estilo \LaTeX\ weciq.cls para preparar um artigo para submissão ao \weciqwcq. O resumo deve conter no máximo 100 palavras.
\end{resumo}
\begin{chave}
Modelo de artigo, \LaTeX, \weciqwcq.
\end{chave}
%% ---------------------------------------------


\begin{abstract}
This document is an example of how to use the \LaTeX\ style weciq.cls to prepare a paper for submission to \weciqwcq. The abstract must have at most 100 words.

For papers written in English, please drop both sections above (Resumo and Palavras-Chave).
\end{abstract}
\begin{keywords}
Paper template, \LaTeX, \weciqwcq.
\end{keywords}


\section{Introdu\c{c}\~{a}o}


 O VIII Workshop-Escola de Computação e Informação Quântica \& VIII Workshop de Computação Quântica - UFSC, está sendo realizado pela Universidade Federal de Santa Catarina com organização compartilhada com a UFCA, UFPE, CEFET-RJ, UFSM, UFRJ e PUC-RJ. 

\subsection{Sobre o \weciqwcq}

O \weciqwcq é um evento nacional com a participação de palestrantes nacionais e internacionais que contempla as áreas temáticas de Ciência e Tecnologias Quânticas, a saber, Computação Quântica, Comunicação Quântica e Metrologia Quântica.

\section{Figuras e Tabelas}
A Tabela \ref{tab:tabela} é apenas um exemplo \cite{ref2}.
\begin{table}[htb]
\caption{\label{tab:tabela}O \textit{caption} vem antes da tabela.}
\begin{center}
{\tt
\begin{tabular}{|c||c|c|c|}\hline
&title page&odd page&even page\\\hline\hline
onesided&leftTEXT&leftTEXT&leftTEXT\\\hline
twosided&leftTEXT&rightTEXT&leftTEXT\\\hline
\end{tabular}
}
\end{center}
\end{table}

A Figura \ref{fig:figura} é apenas um exemplo \cite{ref2}.

\begin{figure}[hbt]
\begin{center}
\setlength{\unitlength}{0.0105in}%
\begin{picture}(242,156)(73,660)
\put( 75,660){\framebox(240,150){}} \put(105,741){\vector( 0, 1){
66}} \put(105,675){\vector( 0, 1){ 57}} \put( 96,759){\vector( 1,
0){204}} \put(105,789){\line( 1, 0){ 90}} \put(195,789){\line(
2,-1){ 90}} \put(105,711){\line( 1, 0){ 60}} \put(165,711){\line(
5,-3){ 60}} \put(225,675){\line( 1, 0){ 72}} \put(
96,714){\vector( 1, 0){204}} \put(
99,720){\makebox(0,0)[rb]{\raisebox{0pt}[0pt][0pt]{a}}}
\put(291,747){\makebox(0,0)[lb]{\raisebox{0pt}[0pt][0pt]{ o}}}
\put(291,702){\makebox(0,0)[lb]{\raisebox{0pt}[0pt][0pt]{ o}}}
\put( 99,795){\makebox(0,0)[rb]{\raisebox{0pt}[0pt][0pt]{ $M$}}}
\end{picture}
\end{center}
\caption{\label{fig:figura}Esta figura \'{e} apenas um exemplo. O
\textit{caption} deve vir ap\'{o}s a figura.}
\end{figure}

\section{Equa\c{c}\~{o}es}

Este é um exemplo de como incluir uma equação no texto.

\begin{equation}\label{eq:exemplo}
    h(t)=\sum_{n=0}^{N-1} \alpha_n\delta(t-\tau_n),
\end{equation}
onde $\alpha_n$ é o $n$-ésimo...

Observe que (\ref{eq:exemplo}) é apenas um exemplo. Existem diferentes formas de incluir equações com múltiplas linhas, como
\begin{equation} \label{eq:exemplo_multiplo}
    \begin{array}{ccl}
        x^2 & = & bx+c\\
        y^2 & = &\beta y+\gamma=0.
    \end{array}
\end{equation}
Eq. (\ref{eq:exemplo_multiplo}) é apenas um exemplo.

\section{Conclusões}
A versão final do artigo aceito para publicação nos Anais do VIII Workshop-Escola de Computação e Informação Quântica \& VIII Workshop de Computação Quântica - UFSC  deve ser enviada, em formato PDF, no máximo até o dia especificado na chamada de trabalhos. O formato do artigo deve ser A4, coluna-dupla, 10pt, lado-único, e possuir no máximo 5 páginas. O Resumo e o \emph{Abstract} devem ter no máximo 100 palavras cada um.

\section*{Agradecimentos}
A Coordenação Técnica do \weciqwcq agradece as coordenações dos WECIQs e WCQs anteriores por todo apoio e material disponibilizado.

\begin{thebibliography}{99}
\bibitem{ref1} L. Lamport, \textit{A Document Preparation System: \LaTeX, User's
Guide and Reference Manual}. Addison Wesley Publishing Company,
1986.
\bibitem{ref2} F. C. Silva e J. J. Sousa, ``Esta referência é apenas um exemplo," ~\textit{Revista de Exemplos}, v. 5, pp. 52--55, Maio
1999.
\end{thebibliography}


\appendix
Inserir as informações referentes aos apêndices aqui.


\end{document}
