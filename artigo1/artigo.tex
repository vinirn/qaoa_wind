\documentclass[12pt,a4paper]{article}
\usepackage[utf8]{inputenc}
\usepackage[portuguese]{babel}
\usepackage{amsmath}
\usepackage{amsfonts}
\usepackage{amssymb}
\usepackage{graphicx}
\usepackage{float}
\usepackage{hyperref}

\title{Otimização Quântica Híbrida (QAOA) para Problema Simples Inspirado no Posicionamento de Turbinas Eólicas: Uma Prova de Conceito}

\author{Marcos A. Santos\\
\textit{Universidade Federal do Rio Grande do Norte}\\
\textit{marcos@exemplo.edu.br}}

\date{\today}

\begin{document}

\maketitle

\begin{abstract}
Este trabalho apresenta uma prova de conceito para aplicação do algoritmo quântico QAOA (Quantum Approximate Optimization Algorithm) em um problema simples inspirado no posicionamento de turbinas eólicas. O modelo simplificado considera efeitos básicos de esteira (wake effects) entre turbinas dispostas em um grid discreto, onde a interferência aerodinâmica é representada por penalidades de proximidade que variam com a direção do vento e distância entre turbinas.

O problema é formulado como otimização QUBO (Quadratic Unconstrained Binary Optimization), utilizando variáveis binárias para representar a presença ou ausência de turbinas em cada posição candidata. A função objetivo maximiza o score energético total enquanto minimiza penalidades de esteira através de um Hamiltoniano de custo personalizado, composto por termos lineares (scores individuais) e quadráticos (interferências entre pares).

A implementação híbrida desenvolvida emprega otimização clássica-quântica variacional, onde um ansatz paramétrico QAOA alterna camadas de evolução temporal pelo Hamiltoniano problema e operadores de mistura. O sistema permite execução tanto em simuladores AER locais quanto em hardware quântico real da IBM (ibm_torino, ibm_brisbane) com transpilação otimizada e gerenciamento automático de recursos computacionais.

Os experimentos abrangem diferentes configurações de grid, utilizando o otimizador clássico COBYLA para ajuste dos parâmetros variacionais γ e β. Os resultados preliminares demonstram convergência satisfatória do algoritmo híbrido, com soluções que respeitam restrições de cardinalidade quando aplicadas. Este trabalho estabelece uma base metodológica robusta para futuras aplicações de algoritmos quânticos NISQ em problemas de otimização combinatória inspirados no contexto de energia renovável, demonstrando a viabilidade prática da computação quântica para essa classe de problemas.
\end{abstract}

\textbf{Palavras-chave:} Computação Quântica, QAOA, Energia Eólica, Otimização Combinatória, IBM Quantum

\section{Introdução}

A otimização do posicionamento de turbinas eólicas é um problema complexo que envolve múltiplos fatores. Neste trabalho, desenvolvemos um modelo simplificado inspirado neste problema real, focando em aspectos fundamentais como interferência entre turbinas em um grid discreto. Esta simplificação permite explorar a aplicabilidade de algoritmos quânticos sem a complexidade de modelos aerodinâmicos completos.

O QAOA representa uma das aplicações mais promissoras da computação quântica na era NISQ (Noisy Intermediate-Scale Quantum), oferecendo vantagem quântica potencial para problemas de otimização complexos.

\section{Metodologia}

\subsection{Formulação do Problema}

O problema simplificado inspirado no posicionamento de turbinas eólicas foi modelado como:
\begin{itemize}
    \item Grid $n \times m$ de posições candidatas
    \item Variáveis binárias $x_i \in \{0,1\}$ indicando presença de turbina
    \item Função objetivo: $\max \sum_i s_i x_i - \sum_{i,j} p_{ij} x_i x_j$
    \item onde $s_i$ é o score de cada posição e $p_{ij}$ são as penalidades de esteira
\end{itemize}

\subsection{Implementação QAOA}

\begin{itemize}
    \item Hamiltoniano de custo: $H_C = -\sum_i s_i Z_i + \sum_{i,j} p_{ij} Z_i Z_j$
    \item Ansatz paramétrico com camadas alternadas de evolução temporal
    \item Otimização clássica dos parâmetros usando COBYLA
    \item Execução híbrida em simulador AER e hardware IBM Quantum
\end{itemize}

\section{Resultados Preliminares}

\subsection{Configurações Testadas}
\begin{itemize}
    \item Grid 3×3 (9 qubits): Convergência em 50 iterações
    \item Grid 10×10 (100 qubits): Execução bem-sucedida em hardware IBM
    \item Shots otimizados: 100 por iteração para reduzir custos
\end{itemize}

\subsection{Validação}
A solução quântica foi validada contra busca exaustiva clássica em grids pequenos, demonstrando concordância com soluções ótimas conhecidas.

\section{Conclusões}

Esta prova de conceito demonstra a viabilidade da aplicação de QAOA para otimização de posicionamento de turbinas eólicas, representando um dos primeiros trabalhos a aplicar computação quântica neste domínio específico da energia renovável. Os resultados indicam potencial para escalabilidade e aplicação em problemas reais de projeto de parques eólicos.

\section{Trabalhos Futuros}

\begin{itemize}
    \item Extensão para grids maiores (>100 qubits)
    \item Incorporação de modelos mais complexos de esteira
    \item Comparação com heurísticas clássicas estado-da-arte
    \item Análise de vantagem quântica em diferentes configurações
\end{itemize}

\bibliographystyle{plain}
\bibliography{referencias}

\end{document}